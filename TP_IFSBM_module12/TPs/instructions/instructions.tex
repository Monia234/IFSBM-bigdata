\documentclass[a4paper, 11pt]{article}

\usepackage[utf8]{inputenc} %% Input encoding: Translates input encodings into LaTex internal language.
\usepackage[T1]{fontenc}    %% Font encoding: Default is 7-bit OT1 while T1 is 8-bit
\usepackage[british]{babel} %% Hyphenation patterns. Default english is US english
\usepackage{hyperref}       %% Produce hypertext links.
\usepackage{datetime}       %% For specifying manual data

%%%% Libraries and parameters for inline R code
\usepackage{color}
\usepackage{listings}
{\lstset{%
  basicstyle=\small\ttfamily,
  commentstyle=\color{red},
  showspaces=false,
  tabsize=2,
  keywordstyle=\color{black},
  showstringspaces=false}
}
\lstset{language=R}                                                                 %% Set R to default language
\lstnewenvironment{rc}[1][]{\lstset{language=R}}{}                                  %% R chunk of code
\newcommand{\ri}[1]{\lstinline{#1}}                                                 %% Short for 'R inline'
\newdate{date}{24}{01}{2020}

\begin{document}
\title{Installation Instructions for Lab 1 and Lab 2 “Big data and predictive models”}
\author{Y. Pradat (MICS, CentraleSupelec),\\L. Verlingue (Institut Gustave Roussy)\\D. Gautheret (Institut Gustave 
Roussy)}
\date{\displaydate{date}}
\maketitle

The following document is a notice for installing the programming tools you will need in order to put into praticse the 
algorithms you saw in the classes. In brief, many different models exists, from the simple linear-regression to the more 
elaborate deep learning models that are parametrized by hundreds of thousands or even millions of parameters.
In order to implement these models, we shall use code that has been developed by other people and which reduces the task 
to a handful of lines of code.  This code is available in the form of \textbf{libraries}, some of which you may already 
have installed and some that you may not.

\section{Git \& GitHub}

\subsection{Git install}

\textbf{Git} is “a distributed version-control system for tracking changes in source code during software development”.  
In other words, it is a very popular tool that allow to efficiently keep track of different versions of files (code, 
text, images, etc.) so that you may revert to \textit{any previous version} without any hassle. It is also extremely 
complete and resourceful for team projects that involve some sort of code collaboration (as for the preparation of the 
labs you will follow). \\ 

\textbf{If you do not have Git installed} on your computer, please install it following the instructions provided 
\href{https://git-scm.com/book/en/v2/Getting-Started-Installing-Git}{here} according to your operating system. Once Git 
is installed, you should have access to a \textit{Git bash} on Windows while a bash (command-line interface) is already 
available on Linux\&MacOS systems.  \textbf{Please check} that the bash command
\begin{lstlisting}[language=bash]
git --version
\end{lstlisting}
gives you the version of Git installed on your machine.

\subsection{GitHub explained}

\textbf{GitHub} is a company that provides hosting services (in the form of a platform www.github.com) with version 
control using Git. If you do not already have a GitHub account, please sign up at \href{www.github.com}{GitHub}. Once 
this is done, you may download (\textit{clone} in Git language) any \textit{repository} available to you on GitHub, i.e 
public repositories or repositories to which you have been granted access.\\ 

The repository for the labs 1 and 2 of “Big data and predictive models” is available at 
\href{https://github.com/gustaveroussy/IFSBM-bigdata}{https://github.com/gustaveroussy/IFSBM-bigdata}. From the bash 
interface, change directory (using the command \texttt{cd}) to whichever location on your machine where you want to have 
the folder for the labs.  Then, \textbf{execute} 
\begin{lstlisting}[language=bash]
git clone https://github.com/gustaveroussy/IFSBM-bigdata
\end{lstlisting}

\section{Installation of an R environment}

In order to follow the next instructions, \textbf{please open} \texttt{Rstudio} and then \texttt{File > New Project > 
Existing Directory > IFSM-bigdata/TP\_IFSBM\_module12/2020/labs > create project}.  This sets the \textit{working 
directory} to \texttt{labs} and creates a \texttt{labs.Rproj} in the folder with settings used by \texttt{Rstudio}.  
Otherwise, use a command-line interface (bash) located at the folder previously specified.
\\

In order to facilitate the management of \texttt{R} libraries, we use 
\href{https://rstudio.github.io/packrat/}{\textbf{packrat}}. In short, packrat makes R projects more isolated, more 
portable and \textit{reproducible}. In the folder \texttt{labs} you have a subfolder named \texttt{packrat} which 
contains the source code (in \texttt{packrat/src}) and binaries (in \texttt{packrat/lib}, \texttt{packrat/lib-R}, 
\texttt{packrat/lib-ext}) of all the libraries we shall need for the labs. 

You can recreate the environment with all the required libraries with \textit{one of the following}
\begin{itemize}
  \item (Rstudio console): \ri{packrat::restore()}
  \item (Bash from the folder labs/ - Please make sure the R interpreter used here is the same as the R interpreter you 
    use in Rstudio):
    \begin{rc}
      R -e ``packrat::restore()''
    \end{rc}
\end{itemize}

\section{Install Keras \& Test}

\textbf{Keras} is an open source \textit{neural network library} written in Python. It is also available in R through a 
wrapper described \href{https://keras.rstudio.com/}{here}. For you to be able to use keras, you shall execute the R 
comand \ri{keras::install\_keras()}. \textbf{This has already been written for you in the example} Rmarkdown 
\texttt{labs/lab\_2/src/TP\_example.Rmd}. Please do \textit{one of the following} to run the example file so that you 
can test your installation:

\begin{itemize}
  \item (Rstudio console): Use the \texttt{Knit} button to generate a html rendering of the Rmarkdown.  \item (Bash from 
    the folder \texttt{labs/}): Run the associated R script.
    \begin{rc}
      Rscript lab_2/src/TP_example.R
    \end{rc}
\end{itemize}
You are more than welcome to look at the code (in the html or in the Rmd) so as to understand how a \textit{neural net 
architecture} may be implemented with \textbf{keras}.

\end{document}

